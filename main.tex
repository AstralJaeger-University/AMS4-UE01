\documentclass[a4paper,11pt]{report}
\usepackage[utf8]{inputenc}
\usepackage[T1]{fontenc}
\usepackage{lmodern}
\usepackage[ngerman]{babel}
\usepackage[margin=20mm, left=20mm, right=10mm, headheight=15pt, includeheadfoot]{geometry}
\usepackage{fancyhdr}
\usepackage{opensans}
\usepackage{titlesec}
\usepackage{tocloft}
\usepackage{titling}
\usepackage{hyperref}
\usepackage{graphicx}
\usepackage{enumerate}
\usepackage{float}
\usepackage{caption}
\usepackage{listings}
\usepackage{minted}
\usemintedstyle{vs}

% Author and subject
\author{Felix Hillebrand}
\newcommand{\subject}{SSD4UE}

\graphicspath{ {./images/} }

% Set default font to OpenSans
\renewcommand*\familydefault{\sfdefault}

% Page style for cover page
\fancypagestyle{cover}{
    \fancyhf{}
    \renewcommand{\headrulewidth}{0pt}
    \renewcommand{\footrulewidth}{0pt}
}

% Page style for main content
\fancypagestyle{main}{
    \fancyhf{}
    \fancyhead[L]{\theauthor}
    \fancyhead[R]{\subject}
    \fancyfoot[C]{\thepage}
    \renewcommand{\headrulewidth}{0.4pt}
    \renewcommand{\footrulewidth}{0pt}
}

\titleformat{\chapter}[block]
  {\normalfont\Large\bfseries} % change \Large to \large or any other size that fits
  {\thechapter}
  {1em}
  {}

  \titlespacing*{\chapter}{0pt}{*4}{*2.5}

% Cover page
\newcommand{\coverpage}{
    \thispagestyle{cover}
    \begin{center}
        % {\includegraphics[height=3cm]{fh-logo.png}}\\[1cm]
        {\LARGE \thetitle}\\[0.5cm]
        {\large \theauthor}\\
        \href{mailto:97hilfel@gmail.com}{97hilfel@gmail.com}\\
    \end{center}
    \tableofcontents
    \clearpage
}

\newcommand{\screenshot}[1]{
    \begin{figure}[H]
        \centering
        \includegraphics[scale=0.375]{#1}
    \end{figure}
}

% Main document
\begin{document}

% Color definitions
\definecolor{LightGray}{gray}{0.9}
\definecolor{DarkGray}{gray}{0.3}

\title{Subject~-~Exercise}
\coverpage

\pagenumbering{roman}
\clearpage
\pagenumbering{arabic}
\pagestyle{main}

\chapter{Python}
    \begin{description}
        \item[Aufgabe:] Vervollständigen Sie folgende Python Funktionen (g ist ein networkx Graph) \hfill
        \item[Lösung:] \hfill \newline % Use hfill to move the figure to the next line
            \begin{minted}[
                frame=lines,
                framesep=2mm,
                baselinestretch=1.2,
                bgcolor=LightGray,
                linenos
                ]{python}
#prints all nodes in g in alphabetical order 
def print_all_nodes(g):
    for node in sorted(g.nodes()):
        print(node)

#adds the nodes start, start+1, start+2,....start+count-1 to the grapgh g 
def add_nodes_(g, start, count):
    g.add_nodes_from([i for i in range(start, start + count)])

#adds the nodes start, start+1, start+2,....start+count-1 to the graph 
#as well as every possible edge between the new nodes 
def add_nodes_connected(g, start, count):
    new_nodes = list(range(start, start + count))
    for i in range(len(new_nodes)):
        for j in range(i + 1, len(new_nodes)):
            g.add_edge(new_nodes[i], new_nodes[j])
            \end{minted}

        \item[Tests:] \hfil \newline
        \begin{minted}[
            frame=lines,
            framesep=2mm,
            baselinestretch=1.2,
            bgcolor=LightGray,
            linenos
        ]{python}
g = nx.Graph()
add_nodes_(g, 0, 5)
print_all_nodes(g)

# Plot graph for better visualization
nx.draw(g, with_labels=True)
plt.show()
        \end{minted}
        Output: 
        \begin{minted}[
            frame=lines,
            framesep=2mm,
            baselinestretch=1.2,
            bgcolor=LightGray,
            linenos
        ]{text}
0
1
2
3
4
        \end{minted}

        \screenshot{notebook/assets/aufgabe_01_nodes.png}

        \begin{minted}[
            frame=lines,
            framesep=2mm,
            baselinestretch=1.2,
            bgcolor=LightGray,
            linenos
        ]{python}
g = nx.Graph()
add_nodes_connected(g, 0, 5)
print_all_nodes(g)

# Plot graph for better visualization
nx.draw(g, with_labels=True)
plt.show()
        \end{minted}
        Output: 
        \begin{minted}[
            frame=lines,
            framesep=2mm,
            baselinestretch=1.2,
            bgcolor=LightGray,
            linenos
        ]{text}
0
1
2
3
4
        \end{minted}
        \screenshot{notebook/assets/aufgabe_01_connected.png}
    \end{description}
\newpage

\chapter{Darstellung}
\begin{description}
    \item[Aufgabe:] Zeichnen Sie den folgenden Graphen \newline
        \begin{math}
            G = (V, E), V = \{a, b, c, d, e\}, E = \{\{a, b\}, \{a, d\}, \{c, e\}, \{b, c\}, \{e, d\}\}
        \end{math}
    \item[Lösung:] \hfill \newline % Use hfill to move the figure to the next line
        \begin{minted}[
            frame=lines,
            framesep=2mm,
            baselinestretch=1.2,
            bgcolor=LightGray,
            linenos
            ]{python}
# Step 1: Create a graph
g = nx.Graph()
g.add_edges_from([('a', 'b'), ('a', 'd'), ('c', 'e'), ('b', 'c'), ('e', 'd')])

# Step 2: Convert to AGraph (Graphviz graph)
a = to_agraph(g)

# Step 3: Draw using AGraph
a.draw("assets/aufgabe_02_graph.png", format="png", prog="dot")

# Step 4: Display the result
display(Image(filename="assets/aufgabe_02_graph.png"))
        \end{minted}
        \screenshot{notebook/assets/aufgabe_02_graph.png}
\end{description}
\newpage

\chapter{Wanderungen, Wege, Pfade}
Gegeben Sei nachfolgende Darstelling eines Graphen mit 8 Knoten



\begin{enumerate}
    \item Geben Sie alle möglichen \textit{Wege} von 1 nach 8 an.
    \item Sind die nachfolgenden Kantenfolgen $W_1$ bis $W_6$ Wanderungen, Wege oder Pfade?
\end{enumerate}

\begin{enumerate}
    % First main item
    \item \label{item:backrefA} \hfill % Use hfill to move the figure to the next line
    \begin{description}
        \item[Aufgabe:] 
         \hfill
        \item[Lösung:] \hfill \newline % Use hfill to move the figure to the next line
            Solution
    \end{description}
\end{enumerate}
\newpage

\chapter{Legal}
Die Ausarbeitung der Aufgabe wurde durch \texttt{OpenAI - GPT-4.5 Turbo}, \texttt{OpenAI - GPT-4.5 Vision}, \texttt{Anthropic -- Claude 3 Opus},  \texttt{Kagi - FastGPT} mit mehreren unterschiedlichen Prompts und Custom Instructions unterstützt.

\end{document}